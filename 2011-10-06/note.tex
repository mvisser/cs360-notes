\documentclass[12pt]{article}
\usepackage{geometry}
\usepackage{amsmath}
\usepackage{amsthm}
\usepackage{amssymb}
\usepackage{mathrsfs}
\usepackage{parskip}
\usepackage{enumerate}
\usepackage{stmaryrd}
\usepackage{listings}
\usepackage{fullpage}
\usepackage{qtree}

\begin{document}

\title{CS 360 Notes}
\author{Matthew Visser}
\date{Oct  6, 2011}
\maketitle

\section{Context-Free Grammars (CFGs)}

Typically used in parsing and sometimes in bio-informatics. These are more
complex than regular expressions. These can be represented using push-down
automata.

An example of this is
\begin{eqnarray*}
	S &\to& subject verb object \\
	S &\to& subject verb \\
	subject &\to& "Dave" "John" "Mary"\\
	verb &\to& "broke" "plays"\\
	object &\to& "the\ guitar"
\end{eqnarray*}

$G$:
\begin{eqnarray*}
	S &\to& A B\\
	A &\to& 0A | \varepsilon\\
	B &\to& B1 | \varepsilon
\end{eqnarray*}

$L(G) = 0^*1^*$

\subsection{Definition}

We can define a CFG as $G = \left< V,T,P,S\right>$
\begin{itemize}
	\item $V$ --- variables
	\item $T$ --- alphabet, $T \cap V = \emptyset$
	\item $P$ --- a finite set of productions

		$V \to \{V,\varepsilon,T\}^*$
	\item $S$ --- start variables
\end{itemize}

A \emph{derivation} is a step expanding a rule, \textit{i.e.},
\begin{eqnarray*}
	S &\to& AB\\
	S &\to& AB1\\
	\hdots\\
	S &\to& AB1111
\end{eqnarray*}


Where $w \in T$, a finite multi-step derivation is
\begin{equation}
	S \Rightarrow^* w
\end{equation}
and a $k$-step derivation is
\begin{equation}
	S \Rightarrow^kw
\end{equation}

\textbf{Base Case}:

Any string of terminals and variables $\alpha$
\begin{equation}
	\alpha \Rightarrow^*\alpha
\end{equation}

\textbf{I.C.}:

If $\alpha \Rightarrow^* \beta$ and $\beta \Rightarrow \gamma$ show $\alpha
\Rightarrow^* \gamma$

\textbf{Leftmost Derivation}: is 
\begin{equation}
	\alpha \Rightarrow^*_{lm} \gamma
\end{equation}

\textbf{Rightmost Derivation}: is 
\begin{equation}
	\alpha \Rightarrow^*_{rm} \gamma
\end{equation}

\begin{enumerate}
	\item $\alpha \Rightarrow^*_{lm} \gamma \iff \alpha \Rightarrow^*_{rm} \gamma $
	\item $\alpha \Rightarrow^* \gamma \iff \alpha \Rightarrow^*_{rm} \gamma $
\end{enumerate}

$G$, the language of $G$
\begin{equation}
	L(G) = \{w \in T^*\ |\ S \Rightarrow^*w \}
\end{equation}

Consider
$S \to 0S1 | S1 | \varepsilon$. $L(G) = \{0^x1^{x+y}\ |\ x,y\ge0\}$

How do we show that the language of $G$ is actually the language?  We do this by
induction. We want to show $L = L(G)$:
\begin{enumerate}
	\item $L \subseteq L(G)$
	\item $L(G) \subseteq L$
\end{enumerate}

\subsection{Proving the Language of a CFG}

Induction on $|w|$.
\textbf{Base Case}: 

$|w| = 0 \implies w = \varepsilon$

$S \to \varepsilon$

\textbf{Inductive Case}:

All words of length $<n$

$w' \in L(G)$
\[
S \Rightarrow^*w'
\]
Take a word $w$, $|w|=n,\ n>0$
\[
w' = 0^x1^y,\quad x+y<n
\]

\textbf{Case 1}:
Say 
\begin{eqnarray*}
	w &=&  0^{x}1^{x+y},\quad y>0\\
	&=& \underbrace{0^x1^{x+y-1}}_{\text{in $L$}}1
\end{eqnarray*}

By I.H. $S \Rightarrow^* 0^x1^{x+y-1}$
\begin{eqnarray*}
	S &\to& S1\\
	S &\to& 0^x1^{x+y-1}1\\
	S &\Rightarrow^*& w
\end{eqnarray*}

\textbf{Case 2}: $w = 0^x1^x,\quad x>0$.

$L(G) \subseteq L$.

Given $w \in L(G)$ show $w \in L$

Induct on $k$, the number of steps in the derivation of $w$.

\textbf{Base Case}:

1 step: $w = \varepsilon \in L$. $x = y = 0.$

\textbf{I.C.}:

Assume every word in $L(G)$ with a shorter derivation than $w$ is in $L$.

$w = 0w'1$ or $w = w'1$ for some $w'$.

\textbf{Case 1}: $w = 0w'1$

By I.H. $w'\in L\\
\therefore w' = 0^{x}q^{x+y}\\
\therefore w = 0^{x+1}1^{x+1+y}\\
\therefore w \in L \because x\ge0,y\ge0$

\textbf{Case 2}: $w = w'1$

$w' = 0^x1^{x+y}$

$\therefore w = 0^x1^{x+(y+1)} \therefore w \in L $

\subsection{Another Example of Proving Language}
Consider $S \to \varepsilon|0S1|1S0|SS$.

$L = \{w \in \{0,1\}^*\ |\ n_0(w) = n_1(w)\}$

\section{Parse Trees}

\begin{description}
	\item[root] is a variable in the grammar
	\item[internal nodes]  are variables generated by productions
	\item[leaves]  are variables, terminals, or $\varepsilon$
\end{description}

For example, say you have $P \to \varepsilon \ |\ 0 \ |\ 1 \ |\ 0P0 \ |\ 1P1$.

$w = 00100$\dots

\Tree
[
.P
0
[
.P
0
[
.P
1
]
0
]
0
]

\end{document}
% vim: tw=80
