\documentclass[12pt]{article}
\usepackage{geometry}
\usepackage{amsmath}
\usepackage{amsthm}
\usepackage{amssymb}
\usepackage{mathrsfs}
\usepackage{parskip}
\usepackage{enumerate}
\usepackage{stmaryrd}
\usepackage{listings}
\usepackage{fullpage}

\begin{document}

\title{CS 360 Notes}
\author{Matthew Visser}
\date{Oct 11, 2011}
\maketitle

\section{Finishing Regular Languages}

\subsection{Closure Properties}

\textbf{Theorem}: If $L$ is a regular language, then its complement, $L'$ is
also regular.

\textbf{Proof}:

$L'$ is all words in $\Sigma^*$ that are not in $L$.

We can find a DFA that has accepting states where normal states were in the one
for language $L$, and the accepting states of the DFA for $L$ are now normal
states.  This will accept any rejected word. \qed

\textbf{Theorem}: If $L_1$ and $L_2$ are regular languages, then $L_1 \cap L_2 $
is also regular.

\textbf{Proof}:

A word $w \in L_1 \cap L_2 $ iff $w \in L_1$ and $w \in L_2$.

\begin{equation}
	L_1 \cup L_2 = (L'_1 \cup L'_2)'
\end{equation}

Since $L_1, L_2$ are regular, then their complements are. The union of two
complements are regular. The complement of that is also regular. \qed

The new $Q = Q_1 \times Q_2$. The new $\delta: \delta( (s_1,t_1),a) = (s_1,t_1)$
iff $\delta_1(s_1,a) = s_2$ and $\delta_2(t_1,a) = t_2$.

The new start states are the pair of start states.

\subsection{Using Closure Properties}

Is the language accepted by a DFA $M$, $L(M)$, infinite?

This is if it has a cycle. You must be able to reach an accepting state from the
cycle.

\textbf{Lemma}: If $M$ is a DFA with $n$ states, then the language $L(M)$ is
infinite iff $\exists x\in L(M)$ such that $n \le |x| \le 2n$.

\textbf{Proof}: If $L(M)$ is infinite, then there is a word $x$ accepted by the
machine $M$ such that $n \le |x| \le 2n$.

If the language is infinite, the machine must have a cycle with length less than
or equal to $n$. We can take the cycle as many times as needed. We can trim a
word by removing cycles, so that it has length satisfying the bounds.

In the other direction\dots If $\exists x \in L(M):\ n \le |x| \le 2n$ then the
$L(M)$ is infinite. Suppose $x \in L(M)$ and $n \le |x| \le 2n$. Using the
pumping Lemma, the word $x = uvw$ so that $uv^yw$ is a subset of $L(M)$. So
$L(M)$ is infinite. \qed

Given two DFAs $M_1$ and $M_2$ 
\begin{itemize}
	\item is $L(M_1) \cap L(M_2)$ empty? Make an intersection machine like the
		one we created for the intersection closure. Check to see if there are
		any states that accept.
	\item is $L(M_1) \subseteq L(M_2)$? Make a machine that has $L(M)'\cap
		L(M_1)$. The only case where the intersection is empty is where $L(M_1)
		\subseteq L(M_2)$.
	\item $L(M_1) = L(M_2)$? Given the two DFAs, check that $L(M_1) \subseteq
		L(M_2)$ and $L(M_2) \subseteq L(M_1)$.
\end{itemize}

\section{Parse Trees}

\textbf{Formal Definition}

Given a grammar $G$, a tree is a \emph{parse tree} if the following is true:
\begin{itemize}
	\item The root of the tree is a variable in $G$.
	\item The interior nodes are labeled by \emph{variables} in $G$.
	\item The leaves are labeled by variables or terminals of $G$.
	\item If there is an internal node labeled $A$, whose children are labelled
		$B_q,B_2,\dots,B_k$ then there must be a production $A \to B_1
		B_2\dots B_k$ in $G$.
\end{itemize}

Any subtree of a parse tree is also a parse tree except for leaves that are
terminals.

Given a parse tree, the \emph{yield} of the parse tree is the concentration of
the symbols at the leaves of the tree, in order from left to right.

\emph{Full parse trees} are trees whose root is the start symbol, and for which
all levels are labelled with terminals or $\varepsilon$.  In these trees, the
yield  corresponds to a word in the grammar.

\subsection{Left-most Derivations}

If we have a leftmost derivation from $S$ to a string $\alpha$ then build a
parse tree starting with $S$ and at each step add children to the leftmost leaf
of the tree labelled with a variable.


\end{document}
% vim: tw=80 
