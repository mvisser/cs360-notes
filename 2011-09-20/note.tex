\documentclass[12pt]{article}
\usepackage{geometry}
\usepackage{amsmath}
\usepackage{amsthm}
\usepackage{amssymb}
\usepackage{mathrsfs}
\usepackage{parskip}
\usepackage{enumerate}
\usepackage{stmaryrd}
\usepackage{listings}
\usepackage{fullpage}

\begin{document}

\title{CS 360 Notes}
\author{Matthew Visser}
\date{Sep 20, 2011}
\maketitle

\renewcommand{\hat}{\widehat}

\textbf{A1 is due on Friday Sept 30\textsuperscript{th} at 1400h.}

\section{NFAs}

\subsection{Extended transition function for NFAs}

\begin{equation}
    \hat{\delta}(q,x) =  Q
\end{equation}
Where $Q$ is the set of states you can get to from $q$ on the input $x$.

\begin{equation}
    \hat{\delta}: Q \times \Sigma^* \to 2^Q
\end{equation}

Base case \begin{equation}
    \hat{\delta}(q,\varepsilon) = \{q\}
\end{equation}

We can't use the same extended transition function as DFAs because
$\hat{\delta}$ returns a set.

We then use:
\begin{equation}
    \bigcup_{p \in \hat{\delta}(q,x)} \hat{\delta}(p,a)
\end{equation}

\section{Accepting}

An NFA $M$ accepts a word $x$ if $\hat{\delta}(q_0,x) \cap F \neq \emptyset$.
That is, from the sets that we can reach from the input string, there is at
least one that is a final state.

\section{A Theorem}

\textbf{Theorem:} Let $L$ be a language that is accepted by an NFA. Then $L$ is
accepted by a DFA.

\textbf{Proof:}

Given an NFA $N$, we'll construct a DFA, $D$. Then we must show that they have
the same language. Every word in the language of $N$ is also in the language of
$D$ and \textit{vice versa}.

For every subset of states in $N$, we create a state in $D$.

Given NFA $N = \left<Q_N, \Sigma, \delta_N, q_0, F_N\right>$, we construct a new
DFA, $D = \left<Q_D, \Sigma, \delta_D, \{q_0\}, F_D\right>$ with these
parameters.

\[Q_D: 2^{Q_N}\] The set of states in $D$ is the set of all subsets of $Q_N$.

\[F_D: \{S \in Q_D | S \cap F_N \neq \emptyset\}\]
States in $D$ that correspond to a set of states in $N$ with at least one
accepting state.

\[\delta_D = (S,a) = \bigcup_{p \in S} \delta_N(p,a)\]
The next state in $D$ is the subset of states in $N$ that is the union of all
states reachable from the subset $S$ on input $a$.

Must show: $x \in L(N) \iff x \in L(d)$ -- that is,\\
$\hat{\delta_N}(q_0,x)\cap F_N \neq \emptyset$ $\iff \hat{\delta_D}(\{q_0\},x) \cap
F_D \neq \emptyset$.

It suffices to show
\[
\hat{\delta_N}\left(q,x\right) = \hat{\delta_D}\left(\{q\},x\right)
\]

We must prove the above by induction on $|X|$.

Base case: $|X| = 0$, so $x = \varepsilon$.

\[\hat{\delta_N}(q,\varepsilon) = \{q\} = \hat{\delta_D}(\{q\},\varepsilon)\]

\textbf{Lemma:} Let $L$ be a language accepted by a DFA, then $L$ is accepted by
an NFA.

\textbf{Proof:} If we allow the DFA's transition function to return a set
instead of just one element, it is now an NFA. This is just a trivial change to
the transition function.

\end{document}
% vim: tw=80
