\documentclass[12pt]{article}
\usepackage{geometry}
\usepackage{amsmath}
\usepackage{amsthm}
\usepackage{amssymb}
\usepackage{mathrsfs}
%\usepackage{parskip}
\usepackage{enumerate}
\usepackage{stmaryrd}
\usepackage{listings}
\usepackage{fullpage}

\newtheorem{thm}{Theorem}
\theoremstyle{definition}
\newtheorem{exmpl}{Example}
\begin{document}

\title{CS 360 Notes}
\author{Matthew Visser}
\date{Nov 17, 2011}
\maketitle

\section{Turing Machines}

\subsection{The Church-Turing Thesis}

Turing machines are equivalent to any reasonable computation model \textit{i.e.}
capture what can be computed by a machine. We can't, however, prove this.

\subsection{Non-Determinism in Turing Machines}

A non-deterministic TM can have multiple values for $\delta(q,a)$.

\begin{itemize}
	\item There must be a finite number of entries.
	\item We know that $(q,yx) \vdash (p,wz)$ if \emph{one} of the transitions in
		$\delta(q,x)$ gets us to the second configuration.
	\item The non-deterministic machine accepts on input $x$ if a valid
		computation exists that brings us from the starting configuration to an
		accepting configuration (in a final state).
\end{itemize}

\begin{exmpl}[Non-Deterministic TM]
	\begin{equation}
		L = \{ ww | w \in \{a,b\}^*, w \neq \varepsilon \}
		\label{eq:ex1Lang}
	\end{equation}

	\begin{itemize}
		\item ``Guess'' the middle position, and then just check.
	\end{itemize}

	Some notation: $(\underbrace{a}_{\text{Left head
	before}},b)/(\underbrace{c}_{\text{Left head
	after}},d),\underbrace{R}_{\text{Left head shift}}L$

	FA left to do later.
\end{exmpl}

\begin{thm}
	If $L$ is the language accepted by a non-deterministic Turing machine $M$,
	there exists a deterministic Turing Machine $M'$ that also accepts $L$.
	\begin{proof}
		We can convert a non-deterministic TM so that at any state we only have
		two choices.

		The deterministic Turing machine just walks through every possible path
		in the non-deterministic machine, $M$. If it finds an accepting path, it
		accepts, otherwise, it keeps going until it has no more paths to try, or
		never terminates.  
	\end{proof}
\end{thm}

\section{Computability}

\subsection{Definitions}

There are different types of languages that we can compute using a Turing
Machine.

A language $L$ is \emph{decidable} or recursive if \dots
\begin{itemize}
	\item a Turing machine, $M$, such that $L(M)=L$ and $M$ halts on all input.
\end{itemize}

A language $L$ is \emph{recursively enumerable}
\begin{itemize}
	\item There exists a Turing machine $M$ such that $L(M)=L$.
\end{itemize}

Set $S$ and $T$ are of \emph{equal cardinality} if $\exists$ a bijection $f:S
\to T$.

$S = \{\text{even integers}\}, T = \{\text{all integers}\}, f(x) = x/2$.

A set $S$ is \emph{countably infinite} when it has the same cardinality as
$\mathbb{Z}$ or any infinite subset of it.


\end{document}
% vim: tw=80
