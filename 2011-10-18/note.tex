\documentclass[12pt]{article}
\usepackage{geometry}
\usepackage{amsmath}
\usepackage{amsthm}
\usepackage{amssymb}
\usepackage{mathrsfs}
\usepackage{parskip}
\usepackage{enumerate}
\usepackage{stmaryrd}
\usepackage{listings}
\usepackage{fullpage}

\begin{document}

\title{CS 360 Notes}
\author{Matthew Visser}
\date{Oct 18, 2011}
\maketitle

\section{Midterm}

\textbf{Where and When}

\begin{tabular}{lp{5cm}}
	Date: & Thursday, Oct 20\\
	Time: & 2:30 - 3:50\\
	Room: & MC 4045
\end{tabular}

\textbf{Office Hours}

\begin{tabular}{lll}
	\textbf{T.A.} & \textbf{Hours} & \textbf{Room} \\
	Soumojit & Wed 12-1 & DC 2302C\\
	Jalaj & Wed 3-4 & DC 3324\\
	Rita & Thur 12-1 & DC2515\\
\end{tabular}

\subsection{What to Study}

Everything up to and not including context-free languages, especially
\begin{itemize}
	\item Pumping Lemma ***
	\item Closure  properties
	\item Constructing DFAs, NFA,s $\varepsilon$-NFAs, regular expressions.
\end{itemize}

\section{Regular Language Enumeration}

Three new Algorithms for Regular Language Enumeration.

Given an NFA and an integer $n$, find the lexographically minimal word of length
$n$ accepted by the NFA.

\section{Context Free Grammar Languages}

$G: S \to S+S\ |\ S*S\ |\ (S)\ |\ a$

$G'$ is defined as
\begin{eqnarray*}
	E &\to& E + T \ |\  T\\
	T &\to& T * F \ |\  F\\
	F &\to& (E)\ |\ a
\end{eqnarray*}

\textbf{Theorem}: $L(G)=L(G')$, and $G'$ is unambiguous.
\begin{proof}
	We will show that $L(G) \subseteq L(G')$ and $L(G') \subseteq L(G)$.

	First we show that $L(G') \subseteq L(G)$ by induction on the length of $x$.
	If $x \in L(G')$ then $x \in L(G)$.

	\begin{description}
		\item[\textbf{Base Case}]
			
			$|x|+1$ Then we have the rule $x \to a$ applied.
		\item[\textbf{I.C.}] 

			$\forall w$ such that $|x|<|x| \in L(G')$ $w$ also occurs in $L(G)$.
			There are three cases here.
			\begin{enumerate}
				\item $x$ derives from the rule $E \to E + T$

					We then have $x = y + z$ where $E \to^* y$ and $T \to^* z$.
					This implies that we can derive $z \in G'$ using $E \to
					T\to^* z$
					This implies that we can derive $y \in G'$ using $E$. The
					lengths $|y|,|z|<|x|$ then $y,z \in L(G)$ by the induction
					hypothesis.

					We can derive $x$ in $G$ by using $S \to S + S$ since we
					know $S \to^*y$ and $S \to^*z$.

				\item $x$ derives from the rule $E \to T \to T * F$

					Then $x = y*z$, where $T \to^*y, F \to^*z$.

					So $y$ and $z$ are shorter than $x$, so $y,z \in L(G)$ by
					the induction hypothesis. So $S \to^*y$ and $S \to^*z$, so
					we use the rule $S \to S*S$ to generate the word $x$.

				\item $x$ derives from the rule $E \to T \to F \to (E)$

					Follow same principle as cases 1-2.
			\end{enumerate}
	\end{description}

	Now we show that $L(G) \subseteq L(G')$. We have three cases:
	\begin{enumerate}
		\item $x$ has a derivation of the form $S \to (S) \to^* (y)$
		\item $x$ has no such derivation $S \to (S) \to^* (y)$, but does have a
			derivation of the form $S \to S+S\to^*y+z$

			We want to show that $y,z \in L(G')$ We want to use the rule $E \to
			E + T$ in $G'$.

			Choose $z$ to be $x$ after the rightmost + sign outside of brackets.
			$E \to T \to^*z$. So to derive in $G'$, $E \to E + T \to^*x$.
		\item The only derivation for $x$ starts by using the rule $S \to S*S$.
	\end{enumerate}
	
\end{proof}

\end{document}
% vim: tw=80
