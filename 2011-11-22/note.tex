\documentclass[12pt]{article}
\usepackage{geometry}
\usepackage{amsmath}
\usepackage{amsthm}
\usepackage{amssymb}
\usepackage{mathrsfs}
\usepackage{parskip}
\usepackage{enumerate}
\usepackage{stmaryrd}
\usepackage{listings}
\usepackage{fullpage}

\newtheorem{thm}{Theorem}
\theoremstyle{definition}
\newtheorem{exmpl}{Example}
\newtheorem{defn}{Definition}

\begin{document}

\title{CS 360 Notes}
\author{Matthew Visser}
\date{Nov 22, 2011}
\maketitle

\section{Church-Turing Thesis}

A set $S$ is \emph{countable} if $S$ has the same cardinality as a subset of
$\mathbb{Z}$.

A set is \emph{uncountable} if it is not countable.

$S = \mathbb{R} \cap [0,1]$ $S$ is ouncountable. By way of contradiction, assume
$S$ is uncountable.

\subsection{Representing TMs using Finite Sequences of 0s and 1s}

$\delta: Q \times \Gamma \to Q \times \Gamma \times \{L,R\}$. We can convert the
transitions to a table.

Every turing machine can be represented as a finite string of 0's and 1's. Call
this representation $f(M)$.

Since we can represent turing machines as a finite string of  0's and 1's, there
are countably many turing machines.

\subsection{How Many Languages}

$L \subseteq \{a\}^* = \{\varepsilon,a,aa,aaa,\dots\}$

How many languages are there that are subsets of $\{a^*\}$?

Given any $L \subseteq \{a\}^*$, we can represent $L$ as a real number between 0
and 1 in binary, but setting the $i^{\text{th}}$ digit after. The decimal point
to 1 iff $a^i \in L$, so there are uncountably many languages.

Since there are uncountably many languages, but countably many turing machines,
there does not exist a turing machine to accept every lanuage.

\subsection{Examples}

\begin{defn}
	A language $L$ is recursively enumerable if $\exists T \text{ s.t. } L(T) =
	L$.
\end{defn}

\begin{defn}
	A language $L$ is decideable if $\exists T \text{ s.t. } L(T) =
	L$ and $T$ halts on all inputs.
\end{defn}

$L_{SA} = \{x|x \text{ is the identifier for a TM that accepts $x$}\}$

$L_{NSA} = \{x | x \text{ is the identifier of some TM that does not accept
istelf}\}$.

\begin{thm}
	$L_{NSA}$ is not the language of any turing machine.

	\begin{proof}

		Pick any TM, $M$. Well show that $L(M) \neq L_{NSA}$. By way of
		contraditction, assume $L(M) = L_{NSA}$.
		\begin{enumerate}
			\item $M$ accepts $f(M)$. Is $f(m) \in L_{NSA}$? No. So $M$ does not
				accept $F(M)$.
			\item $M$ does not accept $F(M)$. Is $F(M) \subset L_{NSA}$? Yes.
				Because $L(M) = L_{NSA}$ then $M$ should accept $f(M)$.
		\end{enumerate}

		So $L(M) \neq L_{NSA}$.

	\end{proof}
\end{thm}


\end{document}
% vim: tw=80
