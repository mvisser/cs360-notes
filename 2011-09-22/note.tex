\documentclass[12pt]{article}
\usepackage{geometry}
\usepackage{amsmath}
\usepackage{amsthm}
\usepackage{amssymb}
\usepackage{mathrsfs}
\usepackage{parskip}
\usepackage{enumerate}
\usepackage{stmaryrd}
\usepackage{listings}
\usepackage{fullpage}

\begin{document}

\renewcommand{\hat}{\widehat}

\title{CS 360 Notes}
\author{Matthew Visser}
\date{Sep 22, 2011}
\maketitle

\section{NFAs}
\subsection{From last class\dots}

Must show $x \in L(N) \iff x \in L(D)$. That is,
\[
\hat{\delta_N}(q_0,x)\cap F_N \neq \emptyset \iff \hat{\delta_D}(\{q_0\},x)\cap
F_D \neq \emptyset 
\]

Suffices to show that
\[
\hat{\delta_N}(q,x) = \hat{\delta_D}(\{q\},x)
\]
for any state $q$.

\subsection{Proof}

\textbf{Base Case:}

$|x| = 0,\quad x = \varepsilon$.
\[
\hat{\delta_N}(q,x) = \{q\} = \hat{\delta_D}(\{q\},x)
\]

\textbf{Inductive Step:}

Let $x = ya$. Assume the result holds for $y$, that is
\[
\hat{\delta_N}(q,y) = \hat{\delta_D}(\{q\},y)
\]

Show 
\begin{eqnarray*}
    \hat{\delta_N}(q,x) &=&  \hat{\delta_D}(\{q\},x)\\
    \hat{\delta_N}(q,ya) &=&  \hat{\delta_D}(\{q\},ya)\\
    \hat{\delta_D}(\{q\},ya) &=&  \delta_D(\hat{\delta_D}(\{q\},y), a)\\
    \hat{\delta_N}(q,ya) &=&  \hat{\delta_N}(\hat{\delta_N}(q,y), a)\\
\end{eqnarray*}

By induction hypothesis:
\[
\hat{\delta_D}(\{q\},ya) =  \delta_D(\hat{\delta_N}(q,y), a)\\
\]

By definition of $\hat{\delta_D}$:
\[
\delta_D(\hat{\delta_N}(q,y), a) = \bigcup_{p \in \hat{\delta_N}(q,y)}
\delta_N(p,a)
\]
Which is $\hat{\delta_n}(q,ya)$.


\section{$\varepsilon$-NFAs}

\subsection{Definition}

Still has
\begin{itemize}
    \item $\Sigma$
    \item $Q$
    \item $\delta$
    \item $F$
    \item $q_0$
\end{itemize}

$\delta : Q \times \Sigma \cup \varepsilon \to 2^Q$

Takes a state and character or $\varepsilon$ and maps to a set of states.

\subsection{$\varepsilon$ Closure}

$\hat{\delta}(q,y)$: states we get to after processing $y$ at state $q$ and
having $\varepsilon$-transitions.

$E_{close}(p)$ is all states reachable from $p$ only using $\varepsilon$-transitions.

We define it as:
\begin{itemize}
    \item $p \in E_{close}(p)$
    \item If $q \in E_{close}(p)$, then so are all states in $\delta(q,\varepsilon)$.
\end{itemize}

Given a set of states, $S$, the enclosure of the set is
\[E_{close}(S) = \bigcup_{s \in S} E_{close}(s)\]

\subsection{The Extended Transition Function of $\varepsilon$-NFAs}

The extended transition function for $\varepsilon$-NFA is defined as:

\textbf{Base Case:} $\hat{\delta}(q, \varepsilon) = E_{close}(q)$

\textbf{Inductive Case:}

Suppose $w = xa, a \in \Sigma$.
\[
\hat{\delta}(q,w) = E_{close}\left(\bigcup_{p \in \hat{\delta}(q,x)}
\delta(p,a)\right)
\]

\subsection{Other Stuff}

THe language of an $\varepsilon$-NFA is
\[
L = \{ x \in \Sigma^*\ |\ \hat{\delta}(q_0,x) \cap F \neq \emptyset\}
\]

\subsection{Proving Equality with DFAs}

\textbf{Theorem:} Given an $\varepsilon$-NFA $E$, there exists a DFA $D$ such
that $L(F) = L(D)$.

\textbf{Lemma:} The converse of the above theorem is true, and proven trivially.

\textbf{Proof Of Theorem:} Given an $\varepsilon$-NFA, $E$, we need to construct a
DFA so that the DFA accepts the same language as $E$.

\[
E = \left< \Sigma, \delta_E, F_E, Q_E, q_0 \right>
\]

Define the DFA,
\[
D = \left< \Sigma, \delta_D , F_D ,Q_D, \{q_0\} \right>
\]
Where
\begin{itemize}
    \item $\Sigma$ is the same.
    \item $Q_D = 2^Q_E$
    \item $\delta_D(S,a) = \bigcup_{s \in S} E_{close}(\delta_E(s,a))$
    \item $F_D =  \{ S \in Q_D\ |\ S \cap F \neq \emptyset \}$
\end{itemize}

Follow the same type of proof as for NFAs.

\section{Regular Expressions}

Language $L$ is \textit{regular} if one of the following is true:
\begin{itemize}
    \item $L$ is the empty language. $L = \emptyset$.
    \item $L$ is the empty language that has the empty word. $L =
        \{\varepsilon\}$.
    \item $L = \{a\}$ for some $a \in \Sigma$.
    \item $L = L_1^*$ for some regular language $L_1$.
    \item $L = L_1L_2$ for some regular languages $L_1,L_2$.
    \item $L = L_1 \cup L_2$ for regular languages $L_1,L_2$.
\end{itemize}

The Kleene Star is defined as \[L_1^* = \bigcup_{i=1}^{\infty}L_1^i\]

\subsection{Theorem 1}

\textbf{Theorem:} All one-word languages are regular.

\textbf{Proof:} Let $L = \{ w \}$. Perform induction on $n = |w|$.

\textbf{Base Case 1}: $n = 0$, $L = \{\varepsilon\}$. This is true by definition.

\textbf{Base Case 2}: $n = 1$, $L = \{a\}$. This is true by definition.

\textbf{Induction Hypothesis:} Let $w = xa$. Assume $L_1 = \{x\}$ is regular. We
need to show that $L = \{xa\}$ is also regular.

Let $L_2 = \{a\}$. This is regular by the induction hypothesis. Let $L =
L_1L_2$. $L$ is regular by definition.


\subsection{Theorem 2}

\textbf{Theorem:} If $L$ has a finite number of words, $L$ is regular.

\textbf{Proof:} By induction on $|L|$.

\textbf{Base Case 1:} $L = \emptyset$. This is true by definition.

\textbf{Base Case 2:} $L = \{\varepsilon\}$. This is true by definition.

\textbf{Base Case 3:} $L = \{a\}$. This is true by definition.

\textbf{Inductive Step:} Suppose all finite languages with $k$ words are
regular. Prove that all languages with exactly $k+1$ words are regular.

Consider language $L_1, |L_1| = k$ and $L_2 = \{a\}$.

Then $L = \{ w_1,w_2,\dots,w_k+1\} = \{w_1,w_2,\dots,w_k\} \cup
\underbrace{\{w_{k+1}\}}_{\text{Regular by base case}$

\end{document}
% vim: tw=80
