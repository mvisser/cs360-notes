\documentclass[12pt]{article}
\usepackage{geometry}
\usepackage{amsmath}
\usepackage{amsthm}
\usepackage{amssymb}
\usepackage{mathrsfs}
\usepackage{parskip}
\usepackage{enumerate}
\usepackage{stmaryrd}
\usepackage{listings}
\usepackage{fullpage}

\newtheorem{thm}{Theorem}

\begin{document}

\title{CS 360 Notes}
\author{Matthew Visser}
\date{Dec  1, 2011}
\maketitle

\section{Rice's Theorem}
\begin{thm}
	Let $P$ be a property held by some but not all recursively enumerable
	languages. Then the language
	\begin{equation}
		L_p = \{f(M) | L(M) \text{ satisfies } P \}
	\end{equation}
	is undecidable.
\end{thm}

Property $P$ can be the property of being infinite. Then
\begin{equation}
	L_p = \{ f(M) : |L(M)| = \infty \}
\end{equation}

Need to show that $\exists T_1$, a turing machine, that satisfies the property and
$\exists T_2$  that fails this property.

$T_1$ the TM that accepts $0^*$ (exists because $O^*$ is regular.

$T_2$ is the TM that accepts $\emptyset$ (exists because it is regular.

Then by Rice's theorem, $L_P = L_{\infty}$ is undecideable.

\begin{proof}[Proof of Rice's Theorem]
	We have * cases:
	\begin{enumerate}
				
		\item Suppose $\emptyset$ doesn't satisfy $P$.
			\begin{itemize}
				\item Pick a recursively enumerable languag $L$ that satisfies
					$P$.
				\item Assume $M_L$ accepts $L$.
				\item Assume by way of contradiction that there exists a TM,
					$M_P$ that decides $L_P$.
			\end{itemize}
		\item Suppose $\emptyset$ does satisfy property $P$.
			\begin{itemize}
				\item Then consider property $P'$, which is the opposite of
					property $P$.
				\item Then the empty set will not satisfy $P'$. We can then use
					the first case for the rest of the proof.
				\item If $P$ is decidable, then $P'$ is as well, which is a
					contradiction, therefore it is undecidable.
			\end{itemize}
	\end{enumerate}
\end{proof}

Given $M_1$ and $M_2$
\begin{itemize}
	\item DO they accept the same language?
	\item $L(M_1) \subseteq L(M_2)$?
	\item Are $L(M_1)$ and $L(M_2)$ disjoint?
	\item Let $M_2$ be a machine that rejects all inputs, so $L(M_2) =
		\emptyset$.
	\item Let $M_3$ be a machine that accepts all inputs.

		Assume by way of contradiction that we have a decider for this. Given a
		TM, $M$, testing if $L(M_1) = L(M_2)$ we can use this to test against
		the empty language.
\end{itemize}<++>

\end{document}
% vim: tw=80
