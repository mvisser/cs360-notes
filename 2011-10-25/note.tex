\documentclass[12pt]{article}
\usepackage{geometry}
\usepackage{amsmath}
\usepackage{amsthm}
\usepackage{amssymb}
\usepackage{mathrsfs}
\usepackage{parskip}
\usepackage{enumerate}
\usepackage{stmaryrd}
\usepackage{listings}
\usepackage{fullpage}

\begin{document}

\title{CS 360 Notes}
\author{Matthew Visser}
\date{Oct 25, 2011}
\maketitle

\section{Taking Up the Midterm}

\subsection{Part 1d}

Note that this is a strict subset. Same as the algorithm for $L_1 \subseteq L_2$
only not equal.

\subsection{Part 2b}

$0^*(1^+0^+)^*1^*0^*$

\subsection{Part 3}

\begin{itemize}
	\item \textbf{Should not have lost a couple of the marks}
	\item Didn't think of the case where we have $0^0$, so if we pumped, then we
		could end up with that. Should add a case where $i>1$.
\end{itemize}

\section{CFG Machines}

What we have is basically an NFA with a stack as well. These machines can
accept CFGs. These machines are called \emph{push-down automata}.

\section{Transitions of Push-Down Automata (PDA)}

Each transition is of the form:
\begin{itemize}
	\item If top letter on the stack is $b$ and next letter in the input is $a$,
		(or $\varepsilon$ if we don't need an input letter) and we are in state
		$q$, then  
		\begin{itemize}
			\item go to state $r$
			\item take $b$ off the top of the memory stack
			\item eat letter $a$ from the input
			\item write word $w$ onto the top of the stack
		\end{itemize}
\end{itemize}

Define a \emph{Push-Down Automata} in a 7-tuple
$M=\left<Q,\Sigma,\Gamma,\delta,q_0,Z_0,F\right>$
\begin{itemize}
	\item $Q$ --- state
	\item $\Sigma$ --- input of alphabet
	\item $\Gamma$ --- finite stack alphabet (often, input $\Gamma = \{Z_0\}\cup
		\Sigma$
	\item $\delta$ --- transition function
	\item $Z_0$ ---stack start letter (bottom of stack character)
	\item $F$ --- finish states
\end{itemize}

We define the transition functions as
\[
d: Q \times (\Sigma \cup \{\varepsilon\}) \times \Gamma \to \text{finite subsets
of } Q \times \Gamma^*
\]
The input is a triple (state, input character, stack character). The output is a
finitely many pairs (state, string over $\Gamma^*$.

\subsection{Instantaneous Description of a PDA}

We have $(q,w,\gamma)$
\begin{itemize}
	\item $q$ is the state that the machine is on
	\item $w$ is what's left of the word
	\item $\gamma$ is what's currently on the stack
\end{itemize}

We use the notation $(q,z,\gamma) \vdash (p,w,\alpha)$ to denote ``produces
in one step''.

Suppose that we are at $(q,z,\gamma)$ and that we can take an
$\varepsilon$-transition. Let $\gamma = X\beta$:
\begin{itemize}
	\item We must follow $\delta(q,\varepsilon,X)$. Suppose that
		$(p,\bar{\alpha}) \in \delta(q,\varepsilon,X)$
	\item This brings us to instantaneous description $(p,Z,\alpha\beta$.
\end{itemize}

\end{document}
% vim: tw=80
