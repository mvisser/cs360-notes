\documentclass[12pt]{article}
\usepackage{geometry}
\usepackage{amsmath}
\usepackage{amsthm}
\usepackage{amssymb}
\usepackage{mathrsfs}
\usepackage{parskip}
\usepackage{enumerate}
\usepackage{stmaryrd}
\usepackage{listings}
\usepackage{fullpage}
\usepackage{graphicx}

\begin{document}

\title{CS 360 Notes}
\author{Matthew Visser}
\date{Sep 15, 2011}
\maketitle

\section{Induction}

\subsection{ Structural}
\begin{itemize}
    \item Binary Trees
    \item Fibonacci numbers
    \item sequences of letters (can think of words as recursive
        structures)
\end{itemize}

\subsubsection{Example}

Let $L$ be the smallest language over $\Sigma = \{+,a,(,)\}$ that satisfies
these rules:
\begin{itemize}
    \item $a \in L$
    \item If $w_1$ and $w_2$ are in $L$, so is ``$(w_1 + w_2)$''.
\end{itemize}

\textbf{Theorem}: No string in $L$ contains ``)(''.

\textbf{Proof}:

Base case: ``a'' does not contain ``)(''.

Assume that words $w_1$ and $w_2$ in $L$ do not contain ``)(''.

We now show that if $w_1$ and $w_2$ satisfy the condition, then $(w_1 + w_2)$
also satisfy the condition. From the form, we know that the first two characters
can't be ``)('' because the first character is ``(''. The last two cannot be
``)('' because the last character is ``)''.

Therefore, by structural induction, the theorem is proven. \qed

\subsubsection{ Strong}
\subsubsection{ Weak}

\section{Deterministic Finite Automata (DFA)}

\begin{itemize}
    \item Given a finite input word, it moves from on input state to another.
    \item Each move is based on one input letter.
    \item At the end of the input it either accepts or rejects.
\end{itemize}

A vital limitation of a DFA is that it cannot look back in the input string. The
only memory it has is in its state.

\subsection{Definition}

\begin{itemize}
    \item $\Sigma$ -- finite input alphabet.
    \item $Q$ -- set of states.
    \item $\delta$ -- transition function.
    \item $q_0$ -- start state.
    \item $F$ -- set accepting states.
\end{itemize}

A DFA machine is defined as:

$M = \left<Q,\Sigma,\delta,q_0,F\right>$

$\delta: \Sigma \times Q \to Q$ \textit{i.e.}\ takes (symbol, state) pairs and
outputs a state.

What does it mean ``$M$ accepts $w$''?
\begin{itemize}
    \item Starting at $q_0$, follow transition function $\delta$ for each letter
        in $w$.
    \item String $w$ accepted by $M$ if at the end of $w$'s letters, we end up
        in a final state.
\end{itemize}

\subsection{Extended Transition Function}

$\hat{\delta}(q,w)$ is defined recursively:
\begin{itemize}
    \item $\hat{\delta}(q,\varepsilon) = q \quad \forall q$
    \item If $|w|>0$, and $W = Xa$, then $\hat{\delta}(q(w) =
        \delta\left(\hat{\delta}(q,X),a\right)$
\end{itemize}

\subsection{Language of an FA}

$L(M)$, all words $M$ accepts. Formally,

$M = \left<Q,\Sigma,\delta,q_0,F\right>$ accepts $w$ exactly when
$\hat{\delta}(q_0,w)\in F$.

$L(M) = \{ w \in \Sigma \text{ s.t. } \hat{\delta}(q_0,w)\in F\}$

\subsection{Example}

$x \equiv 0 \mod 3$

$\Sigma = \{0,1\}$

$Q = \{ q, q_0, q_1, q_2 \}$

$F = \{q_0\}$


\begin{tabular}{l|l|l}
    start state & input & end state \\
    \hline
    $q$         & 0     & $q_0$     \\
    \multicolumn{3}{c}{$\vdots$}
\end{tabular}

\section{Non-Deterministic Finite Automata (NFA)}

These accept if there is at least one path that leads to an accepting state.
Formally, we have
\begin{itemize}
    \item $\Sigma$ -- finite input alphabet.
    \item $Q$ -- set of states.
    \item $\delta$ -- transition function. This is different from a DFA.
    \item $q_0$ -- start state.
    \item $F$ -- set accepting states.
\end{itemize}

$\delta: \Sigma \times Q \to \{ \text{ subsets of $Q$}\} = 2^Q$

$2^Q$ is all subsets of $Q$.


\end{document}
% vim: tw=80
