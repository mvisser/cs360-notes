\documentclass[12pt]{article}
\usepackage{geometry}
\usepackage{amsmath}
\usepackage{amsthm}
\usepackage{amssymb}
\usepackage{mathrsfs}
\usepackage{parskip}
\usepackage{enumerate}
\usepackage{stmaryrd}
\usepackage{listings}
\usepackage{fullpage}

\begin{document}

\title{CS 360 Notes}
\author{Matthew Visser}
\date{Nov 10, 2011}
\maketitle

\section{Turing Machines}

We can represent a turing machine as a finite automaton plus a memory tape. We
can only see one letter of memory at a time.

Formally:
\begin{equation}
	M = \left<Q,q_0,F,\Sigma,\Gamma,\beta,\delta\right>
\end{equation}
where $\Gamma$ is the tape alphabet, and $\beta$ is the blank character for a tape.
$\delta$ is a transition
\begin{equation}
	\delta: Q \times \Gamma \to Q \times \Gamma \times \{L,R\}
\end{equation}
Thie transition function need not be a full function. It takes a current state
and an input letter, then outputs a new state, a letter to write at the current
place in the tape, and a direction to move.

This machine accepts at an accepting state.

\end{document}
% vim: tw=80
