\documentclass[12pt]{article}
\usepackage{geometry}
\usepackage{amsmath}
\usepackage{amsthm}
\usepackage{amssymb}
\usepackage{mathrsfs}
\usepackage{parskip}
\usepackage{enumerate}
\usepackage{stmaryrd}
\usepackage{listings}
\usepackage{fullpage}

\begin{document}

\title{CS 360 Notes}
\author{Matthew Visser}
\date{Sep 27, 2011}
\maketitle

\section{Some Stuff Useful for Assignment}

Given a DFA $M$ and a word $x$, does $M$ accept $x$? Yes, simulate it.

Given a DFA $M$ and a word $x$, is $L(M)$ empty? If a final state is reachable
from a start state, no.

\section{Regular Expressions}

We have the following expressions that are a valid Regular Language.
$\phi,  \{\varepsilon\},\{d\}\ d \in \Sigma, L_1^* L_1L_2, L_1 \cup L_2$

\subsection{Examples}

\begin{enumerate}
    \item  $\Sigma = \{0,1\}$ and $\Sigma^* = \{0,1\}^*$ are equivalent to $(0 + 1)^*$.

    \item Write an expression for even length sequences:
        \[
        ( (0 + 1)(0 + 1))^*
        \]

    \item Write an expression for sequences of length at most 3.
        \[
        (0 + 1 + \varepsilon)
        (0 + 1 + \varepsilon)
        (0 + 1 + \varepsilon)
        \]

    \item Write an expression for sequences with at most 2 zeroes.
        \[
        1^*(0 + \varepsilon)1^*(0 + \varepsilon)1^*
        \]

\end{enumerate}

\subsection{Rules of Regular Expressions}

\begin{enumerate}
    \item $\emptyset \varepsilon = \emptyset = \varepsilon \emptyset$
    \item $\emptyset^* = \varepsilon$
    \item $x + x = x$
    \item $(x^*)^* = x^*$
    \item $x(y + z) = xy + xz$
\end{enumerate}

\subsection{Kleene's Theorem}

\textbf{Theorem:} All regular languages are the language of a DFA, and all DFAs
accept a regular language.

\textbf{Proof:}

We well prove that if a language is regular then there exists a DFA that accepts
it.

It suffices to show that if a language is regular then there exists an
$\varepsilon$-NFA that accepts it. A regular language is either
$\{\varepsilon\}, \emptyset, \{d\}$ where $d \in \Sigma, L_1L_2, L_1^*$, or
$L_1\cup L_2$.

We can write pieces of NFA equivalent to each of these base rules. We then
perform structural induction on the form of a regular language.

\textbf{Base Case}:
For $\{\varepsilon\}$ we can have an NFA with just the accepting state.
For $\emptyset$ we can have an NFA with no accepting state.
For $\{a\}$ we can have an NFA with an edge from the start state to an accepting
state on input $a$.

\textbf{Inductive Step}:

Assume there exists $\varepsilon$-NFAs $M_1$ and $M_2$ that accept regular
languages $L_1$ and $L_2$ respectively.

$L_1\cup L_2$ can be drawn by having a start state with an $\varepsilon$
transition to the start states of the machines for $L_1$ and $L_2$ respectively.

Then $L_1L_2$ can be drawn by removing the accepting state from $L_1$ and
pointing it to the start state of $L_2$ on input $\varepsilon$.

Then $L_1*$ can be drawn by removing the accepting state from $L_1$ and pointing
it to the new start state on input $\varepsilon$. Also allow the start
state to be an accepting state.

The proof for the Kleene start construction is: suppose $x$ is accepted by the
machine that we constructed. Show that $x \in L_1^*$. Since $x$ is accepted by
our machine, then each cycle from the start state to the start state is a
word from $L_1$, so $x \ in L_1^*$.

The proof in the other direction is: assuming $x \in L_1^*$, show that $x$ is
accepted by our machine. $\exists n$ such that $x = w_1w_2\dots w_n,\ w_i \in
L_1,\ 1 \leq i \leq n$ or $x = \varepsilon$. In the second case, $x$ will be
accepted by the start state. In the first case, our machine accepts $w_i$, then
goes to the start state again, and can accept $w_{i+1}$.

\end{document}
% vim: tw=80
