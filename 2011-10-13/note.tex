\documentclass[12pt]{article}
\usepackage{geometry}
\usepackage{amsmath}
\usepackage{amsthm}
\usepackage{amssymb}
\usepackage{mathrsfs}
\usepackage{parskip}
\usepackage{enumerate}
\usepackage{stmaryrd}
\usepackage{listings}
\usepackage{fullpage}

\begin{document}

\title{CS 360 Notes}
\author{Matthew Visser}
\date{Oct 13, 2011}
\maketitle

\section{Midterm}

Covers everything except for CFG's from last class.

\section{Ambiguity in CFGs}

Say we have a grammar $G$. Sometimes there are two different ways to interpret
the grammar.

Consider
\begin{equation}
	G: S \to SS | \texttt{if test:} S | \texttt{if test:} S \texttt{else:} S | foo
	\label{grammar1}
\end{equation}

This is ambiguous. Consider
\begin{verbatim}
if test:
    if test:
        foo
    else:
        foo
\end{verbatim}
and
\begin{verbatim}
if test:
    if test:
        foo
else:
    foo
\end{verbatim}

\subsection{Inherent Ambiguity}

\begin{equation}
	L = \{a^ib^jc^k\ |\ i=j \text{ or } j=k\}
	\label{language1}
\end{equation}

Consider $w = d^ib^ic^i$.

A context-free grammar $G$ is \emph{ambiguous} if there exists a word $w \in
L(G)$ with two different parse trees in $G$.

A Context-free language is \emph{inherently ambiguous} if $\forall G:\ L(G) =
L$, $G$ is ambiguous, \textit{i.e.}, the language has no unambiguous grammar.

\subsection{A Theorem}

\textbf{Theorem}: If $G$ is the grammar
\begin{equation}
	S \to S + S | S * S | (S)|a
	\label{grammar2}
\end{equation}
and $G'$ is the grammar
\begin{equation}
	\begin{split}
	E &\to E + T |T\\
	T &\to T*F|F\\
	F &\to(E)|d
	\end{split}
	\label{grammar3}
\end{equation}
then $L(G) = l(G')$, and $G'$ is unambiguous.

\textbf{Lemmas}
\begin{itemize}
	\item If $T \implies^* X$, then $X$ has no $+$ outside of parenthesis.
	\item If $F \implies^* X$, then $X$ has no + or * outside parenthesis.
\end{itemize}

\textbf{Lemma}: If $E \to^* X$ or $T \to^* X$ or $F \to^*X$, then there is a
unique left-most derivation for that sequence from $E$.

\textbf{Proof}: By induction on $|X|$.

\begin{enumerate}
	\item[\textbf{B.C.}] $|X|=1, X=a$.
		\[ E \to T \to F \to a \]

	\item[\textbf{I.H.}] Assume that all words of length $<n$
		have a unique leftmost derivation.

	\item[\textbf{I.C.}] Pick any word $X$ of length $n$ derived from
		$E,T,\text{ or }F$.

		\begin{enumerate}
			\item[\textbf{Case 1}] There is a + outside of parenthesis in $X$

				We need to start the derivation with
				\[ E \to E + T \]

				So $X = Y + Z$ for some $X,Y,Z$.

				Choose $Z$ to be after the last + in $X$ outside of parenthesis.
				\begin{itemize}
					\item $Y$ can be derived from $E$ by inductive hypothesis
					\item $Z$ can be derived from $T$ by inductive hypothesis,
						since $Z$ has no + outside parenthesis.
				\end{itemize}<++>
			\item[\textbf{Case 2}] There is no + outside of parenthesis, but
				there is * outside parenthesis.

				Start with rule
				\[ T \to T * F \]
				then we have $X = y * z$.

				$z$ must correspond to the string right after the last *. By the
				inductive hypothesis, there is only one way to derive $y$ from
				$T$ and to derive $z$ from $F$.

			\item[\textbf{Case 3}] There is neither a + or * outside of
				parenthesis.

				Start with rule
				\[F \to (E)\]
				So $X = (y)$ where $E \to^* y$ and $|y|<n$. Therefore, but the
				I.H., there is a unique derivation.
		\end{enumerate}
\end{enumerate}
\hfill $\blacksquare$


\end{document}
% vim: tw=80
